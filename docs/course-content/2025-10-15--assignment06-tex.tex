\documentclass{article}
\usepackage{amsmath}
\usepackage{amsfonts}
\usepackage{amssymb}
\usepackage{tikz-cd} % For commutative diagrams

\title{Problem Set 6}
\author{Tufts University \\ Fall 2025 \\ Math 065 \\ Prof. George McNinch}
\date{Due: October 15, 2025}

\begin{document}

\maketitle


\begin{enumerate}

\item
Let $ R $ be a commutative ring (with identity). We write $ 0 $ for the trivial $ R $-module $ \{0\} $.

Consider a diagram $ \mathcal{A} $ of the form:
$$\cdots \rightarrow A_{i-1} \xrightarrow{d_{i-1}} A_i \xrightarrow{d_i} A_{i+1} \xrightarrow{d_{i+1}} A_{i+2} \rightarrow \cdots$$
where for $ i \in \mathbb{Z} $, $ A_i $ is an $ R $-module and $ d_i: A_i \rightarrow A_{i+1} $ is an $ R $-module homomorphism. Then $ \mathcal{A} $ is said to be a \emph{complex} provided that $ d^2 = 0 $; i.e., that for each $ i \in \mathbb{Z} $ we have $ d_i \circ d_{i-1} = 0 $. This implies that $ \text{im} \, d_{i-1} \subseteq \ker \, d_i $.

And the complex $ \mathcal{A} $ is said to be \emph{exact} provided that for all $ i \in \mathbb{Z} $, $ \text{im} \, d_{i-1} = \ker \, d_i $.

Let $ \mathcal{A} $ be a complex:

\begin{enumerate}
    \item For $ i \in \mathbb{Z} $ write $ H^i(\mathcal{A}) $ for the $ R $-module $ \ker d_i / \text{im} \, d_{i-1} $. Show that $ \mathcal{A} $ is exact if and only if $ H^i(\mathcal{A}) = 0 $ for each $ i \in \mathbb{Z} $.

    \item For $ R $-modules $ X, Y, Z $, we view a diagram 
    $$0 \rightarrow X \xrightarrow{f} Y \xrightarrow{g} Z \rightarrow 0$$
    as a complex provided that $ g \circ f = 0 $ by taking $ A_i = 0 $ for $ i \leq 0 $, $ A_1 = X $, $ A_2 = Y $, $ A_3 = Z $, and $ A_j = 0 $ for $ j \geq 4 $ as well as $ d_1 = f $, $ d_2 = g $, and $ d_j = 0 $ for $ j \neq 1,2 $.

    We say that the complex $ 0 \rightarrow X \rightarrow Y \rightarrow Z \rightarrow 0 $ is a \emph{short exact sequence} provided that it is an exact complex. 
    
    Prove that $ 0 \rightarrow X \rightarrow Y \rightarrow Z \rightarrow 0 $ is a short exact sequence if and only if
    (i) $ f $ is injective, (ii) $ \ker(g) = \text{im}(f) $, and (iii) $ g $ is surjective.

    \item Let $ \phi: M \rightarrow N $ be an $ R $-module homomorphism. Show that 
    $$0 \rightarrow \ker \phi \xrightarrow{\iota} M \xrightarrow{\overline{\pi}} \text{im} \, \phi \rightarrow 0$$
    is a short exact sequence, where $ \iota: \ker \phi \rightarrow M $ and $ \pi: M \rightarrow \text{im} \, \phi $ are the inclusion mapping and the quotient mapping, respectively.
\end{enumerate}

\item
Let \( M\), $ N $ be $ R $-modules. Show that there is a short exact sequence
$$0 \rightarrow M \xrightarrow{\iota_M} M \oplus N \xrightarrow{\pi_N} N \rightarrow 0$$
where $ \iota_M: M \rightarrow M \oplus N $ and $ \iota_N: N \rightarrow M \oplus N $ are the inclusion maps, and $ \pi_M: M \oplus N \cong M \times N \rightarrow M $ and $ \pi_N: M \oplus N \cong M \times N \rightarrow N $ are the projections.

\item
For ideals $ I, J \subseteq R $, the \emph{product} of $ I $ and $ J $ is the ideal generated by $ \{xy \mid x \in I, y \in J\} $.

\begin{enumerate}
    \item Prove that $ IJ \subseteq I \cap J $.
    \item If $ P \subseteq R $ is a \emph{prime ideal} and if $ IJ \subseteq P $, prove that either $ I \subseteq P $ or $ J \subseteq P $.
\end{enumerate}

\item
An element $ a \in R $ is said to be \emph{nilpotent} if $ \exists N \in \mathbb{N}, a^N = 0 $.

For an ideal $ I $ of $ R $ and $ n \in \mathbb{N} $ we define the ideal $ I^n $ inductively as follows:

\begin{itemize}
    \item $ I^0 = R $, and 
    \item for $ n > 0, I^n = I \cdot I^{(n-1)} $.
\end{itemize}

An ideal $ I $ is \emph{nilpotent} if $ \exists N \in \mathbb{N}, I^N = 0 $.

\begin{enumerate}
    \item If $ a \in R $ is nilpotent, prove that $ 1 - ab $ is a unit in $ R^* $ for every $ b \in R $, where $ R^* $ is the set of units of $ R $.
    \item Let $ I = \langle a_1, a_2, \ldots, a_m \rangle $ for $ a_i \in R $ be a finitely generated ideal. Prove that if $ a_i $ is nilpotent for all $ i $, then $ I $ is a nilpotent ideal.
\end{enumerate}

\item
Let $ G $ be a group, and let $ R[G] $ be the \emph{monoid algebra} of $ G $. Thus $ R[G] $ is a free $ R $-module with a basis $ \{e(g) \mid g \in G\} $ and the multiplication satisfies $ e(g) e(h) = e(gh) $ for $ g, h \in G $.

\begin{enumerate}
    \item Prove that 
    $$I = \left\{\sum_{g \in G} a_g e(g) \in R[G] \; \bigg| \; \sum_{g \in G} a_g = 0 \right\}$$
    is a two-sided ideal of $ R[G] $ and that the $ R $-algebra $ R[G] / I $ is isomorphic to $ R $. $ I $ is called the \emph{augmentation ideal} of $ R[G] $.

    \item Let $ p $ be a prime number, let $ G = \langle \sigma \rangle \cong \mathbb{Z}/p\mathbb{Z} $ be the cyclic group of order $ p $ (written multiplicatively), and let $ \mathbb{F}_p = \mathbb{Z}/p\mathbb{Z} $ be the field of $ p $ elements. Show that the augmentation ideal $ I $ of $ \mathbb{F}_p[G] $ has a basis consisting of $ \{ e(1) - e(\sigma^i) \mid i = 1, \ldots, p-1 \} $ and show that  $ I $ is a nilpotent ideal.
\end{enumerate}

\item
Let $ R[T] $ be the polynomial ring in a single variable over $ R $.
Recall that for $ f \in R[T] $, $ \langle f \rangle = f \cdot R[T] $ denotes the principal ideal generated by $ f $.

\begin{enumerate}
    \item Let $ f = T^n + a_{n-1} T^{n-1} + \ldots + a_1 T + a_0 $ for $ n \in \mathbb{N} $ and $ a_i \in R $.
    
    Prove that the quotient ring $ R[T] / \langle f \rangle $ is a free $ R $-module with basis $ \{ \overline{T^i} = T^i + \langle f \rangle \mid 0 \leq i \leq n-1 \} $.
    
    \item Prove that $ \mathbb{Z}[T] / \langle 2T \rangle $ is not a free $ \mathbb{Z} $-module. Describe this ring as a $ \mathbb{Z} $-module (i.e., as an abelian group).
\end{enumerate}

\item
A ring $ R $ is said to be a \emph{local ring} if it has a unique maximal ideal.

\begin{enumerate}
    \item Prove that if $ R $ is local with unique maximal ideal $ M $, then every element of $ R \setminus M $ is a unit in $ R $.
    
    \item Conversely, prove that if the set of non-units in $ R $ forms an ideal $ M $, then $ R $ is local with unique maximal ideal $ M $.
    
    \item Prove for a prime $ p \in \mathbb{Z} $ that 
    $$R \subseteq \mathbb{Q}$$ 
    defined by
    $$R = \left\{\frac{a}{b} \mid a,b \in \mathbb{Z}, b \not\equiv 0 \ (\text{mod } p) \right\}$$
    is a local ring with unique maximal ideal $ pR = \langle p \rangle $.
\end{enumerate}

\end{enumerate}

\end{document}
