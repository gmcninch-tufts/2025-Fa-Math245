\documentclass{article}
\usepackage{amsmath}
\usepackage{amssymb}
\usepackage{amsfonts}
\usepackage{tikz}
\usepackage{enumerate}
\usetikzlibrary{arrows}

\title{Problem Set week 11}
\author{Tufts University \\ Math 065 \\ Prof. George McNinch}
\date{Due: November 19, 2025}

\begin{document}

\maketitle
\begin{enumerate}
\item
Let $ A $ be a ring and let $ 0 \to X \xrightarrow{j} Y \xrightarrow{\psi} Z \to 0 $ be a short exact sequence of $ A $-modules.

Prove that the following are equivalent:
\begin{enumerate}[(a)]
    \item There is an isomorphism $ \phi : X \oplus Z \to \tilde{Y} $ such that $ \psi \circ \phi = \pi_2 $ and $ \phi^{-1} \circ j = \iota_1 $ where
    $ \pi_2 : X \oplus Z \to Z $ is given by $ (x,z) \mapsto z $ and $ \iota_1 : X \to X \oplus Z $ is given by $ x \mapsto (x,0) $.
    
    \item There is a section to $ \psi $; i.e., there is an $ A $-module homomorphism $ \sigma : Z \to Y $ such that $ \psi \circ \sigma = \text{id}_Z $.
    
    \item There is a retraction of $ j $; i.e., there is an $ A $-module homomorphism $ \rho : Y \to X $ such that $ \rho \circ j = \text{id}_X $.
\end{enumerate}
The short exact sequence $ 0 \to X \xrightarrow{j} Y \xrightarrow{\psi} Z \to 0 $ is said to be \textit{split exact} if it satisfies these equivalent conditions.

\item
Let $ 0 \to X \xrightarrow{\iota} Y \xrightarrow{\pi} Z \to 0 $
be a short exact sequence of $ A $-modules.

Prove that for an $ A $-module $ W $, the sequence
$$0 \to \text{Hom}_A(W,X) \xrightarrow{f \mapsto f \circ \iota} \text{Hom}_A(W,Y) \xrightarrow{g \mapsto g \circ \pi} \text{Hom}_A(W,Z) \to 0$$
is \textit{exact}.

\textbf{NB}: A sequence of $ A $-modules and $ A $-module homomorphisms is \textit{exact} if for each interior term 
$ S \xrightarrow{f} T \xrightarrow{g} Y $ of the sequence, $ \ker(g) = \text{im}(f) $.

In the sequence above, $ \text{Hom}_A(P,Y) $ is not an interior term.

\textbf{Remark}: The exactness of this sequence for each short exact sequence is often expressed by saying that the functor $ \text{Hom}_A(W,-) $ is \textit{left-exact}.

Thus, your task is to check the exactness of the sequence at the interior terms $ \text{Hom}_A(W,X) $ and $ \text{Hom}_A(W,Y) $.

\item
An $ A $-module $ P $ is said to be \textit{projective} if for every surjective homomorphism of $ A $-modules $ \pi : M \to N $ and every $ A $-module homomorphism $ f : P \to N $, there is an $ A $-module homomorphism $ \alpha : P \to M $ such that $ \pi \circ \alpha = f $.

\begin{enumerate}
    \item Prove that any free $ A $-module is projective.
    \item Prove that a module $ P $ is projective if and only if $ P $ is a direct summand of a free module; i.e., if there is a free $ A $-module $ F $ and an isomorphism of $ A $-modules \( F 
      \cong P \oplus Q \) for some $ A $-module $ Q $.
\end{enumerate}

\item
Prove that the following are equivalent:
\begin{enumerate}
    \item $ P $ is projective.
    \item For every short exact sequence of $ A $-modules $ 0 \to X \to Y \to Z \to 0 $, the sequence of $ A $-modules
    $$0 \to \text{Hom}_A(P,X) \to \text{Hom}_A(P,Y) \to \text{Hom}_A(P,Z) \to 0$$
    is exact.
\end{enumerate}
\textbf{Remark}: Thus $ P $ is projective if and only if the left-exact functor $ \text{Hom}_A(P,-) $ is exact (in the sense that $ \text{Hom}_A(P,-) $ carries short exact sequences to short exact sequences).

\item
Let $ A $ be a commutative ring, let $ B = A \times A $, and let $ \pi : B \to A $ be the projection ring homomorphism given by $ \pi(x,y) = x $.

Using $ \pi $, we can view $ A $ as a $ B $-module. Prove that $ A $ is a projective $ B $-module that is not a free $ B $-module.

\item
Let $ K $ be a field and let $ a,b \in K \setminus \{0,1\} $ with $ a \neq b $.
Let $ q = T(T-a)(T-b) \in K[T] $ and consider the polynomial $ f = S^2 - q \in K[S,T] $.

You proved in the previous homework set that:
\begin{itemize}
    \item $ A = K[S,T] / \langle f \rangle $ is an integral domain. Write $ s,t $ for the images of $ S,T $ in the quotient ring $ A $; thus $ s^2 = q(t) = t(t-a)(t-b) $ in $ A $.
    \item The map $ K[T] \to K[t] $ for which $ T \mapsto t $ is an isomorphism.
    \item $ A $ is a free $ K[t] $-module on a basis $ \{1,s\} $.
\end{itemize}

\begin{enumerate}
    \item Define $ N: A \to K[t] $ by the rule $ N(\alpha) = N(f + gs) = f^2 - g^2 \cdot q(t) $ for $ \alpha = f + gs $ with $ f,g \in K[t] $.
    Then $ N $ is multiplicative (i.e., it is a monoid homomorphism) and $ \alpha \in A $ is a unit in $ A $ if and only if $ N(\alpha) \in K[T]^\times = K^\times $.
    
    \item Write $ \mathfrak{m} = \langle s,t \rangle $ for the ideal of $ A $ generated by $ s $ and $ t $. Show that the ideal $ \mathfrak{m} $ is not principal.
    
    \textbf{Hint:} Suppose to the contrary that $ \alpha = f + gs $ in $ A $ is a generator for $ \mathfrak{m} $, for $ f,g \in K[t] $.
    Then $ \alpha $ divides $ t $ $ \Rightarrow N(\alpha) $ divides $ t^2 $. Now use a degree argument to show that $ g = 0 $.
    
    \item Show that $ \mathfrak{m} $ is a maximal ideal of $ A $. 
\end{enumerate}

\textbf{Remark 1}: Essentially the same arguments show that the ideals $ \langle s, t-a \rangle $ and $ \langle s, t-b \rangle $ are maximal and not principal.

\textbf{Remark 2}: The field of fractions $ F = \text{Frac}(A) = K(s,t) $ of $ A $ is the \textit{field of rational functions} on the \textit{elliptic curve} over $ K $ defined by the cubic equation \( S^2 = T(T-a)(T-b) \).

If we had considered instead the quadratic equation $ S^2 = T(T-a) $, the analogous ideal $ \mathfrak{m} = \langle s,t \rangle $ is principal.

\item
Keep the notations of the previous problem. We are going to show that $ \mathfrak{m} $ is a projective $ A $-module.

\begin{enumerate}
    \item First, explain why the condition that $ \mathfrak{m} $ is not a principal ideal implies that $ \mathfrak{m} $ is not a free $ A $-module.
    
    \item Explain why $ 1 = u(t) + v(t-a)(t-b) $ for some $ u,v \in K[t] $.
    
    \item Show for every $ z \in \mathfrak{m} $ that $ \frac{(t-a)(t-b)z}{s} \in A $; note that \textit{a priori} $ \frac{(t-a)(t-b)z}{s} $ is an element of the field of fractions of $ A $.
    
    \item Consider the surjective $ A $-module homomorphism $ \pi : A^2 \to \mathfrak{m} $ given by $ \pi \begin{pmatrix} x \\ y \end{pmatrix} = xs + yt $.
    Show that $ \sigma : \mathfrak{m} \to A^2 $ given by the rule 
    $$\sigma(z) = \begin{pmatrix} \frac{v(t-a)(t-b)z}{s} \\ uz \end{pmatrix}$$
    is a \textit{section} to $ \pi $ (where $ u,v $ are as in part (b)).
    
    \item Conclude using Problem 1 that $ \mathfrak{m} $ is isomorphic to a direct summand of the free module $ A^2 $.
\end{enumerate}
\end{enumerate}

\end{document}
