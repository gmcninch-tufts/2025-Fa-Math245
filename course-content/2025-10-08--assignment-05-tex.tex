\documentclass{article}
\usepackage{amsmath}
\usepackage{amsfonts}
\usepackage{amssymb}
\usepackage{graphicx}
\usepackage{color}
\usepackage{enumitem}

\title{Problem Set 5}
\author{Tufts University \\ Fall 2025 \\ Math 065 \\ Prof. George McNinch}
\date{Due: October 8, 2025}

\begin{document}

\maketitle



Let $R$ be a commutative ring (with identity).

\begin{enumerate}

\item 
\begin{enumerate}[label=\alph*.]
    \item Let $M$ be an $R$-module, and let $I$ be an index set. For $i \in I$, let $M_i \subseteq M$ be an $R$-submodule. Write $\sum_{i \in I} M_i$ for the sum of submodules $M_i$. Thus $\sum_{i \in I} M_i$ coincides with the \emph{submodule generated by the $M_i$}.

    For a finitely supported function $(*)$ $f:I \to \bigcup_{i \in I} M_i$ for which $f(j) \in M_j$ for each $j \in I$, note that $\Sigma(f) = \sum_{i \in I} f(i)$ is a well-defined element of $M$.

    Prove that $\sum_{i \in I} M_i = \{ \Sigma(f) \mid f \text{ is a finitely supported function satisfying } (*) \}$.

    \item Assume that $M = \sum_{i \in I} M_i$ and that for a finitely supported function $f$ satisfying $(*)$, we have $\Sigma(f) = 0 \implies f(i) = 0$ for each $i \in I$. Prove that $M \cong \bigoplus_{i \in I} M_i$.

    One says in this case that $M$ is the \emph{internal direct sum} of the submodules $M_i$.

    \item Let $X_1, X_2 \subseteq M$ be $R$-submodules. Suppose that $M = X_1 + X_2$ and that $X_1 \cap X_2 = 0$. Prove that $M \cong X_1 \oplus X_2$.

    One says in this case that $M$ is the \emph{internal direct sum} of $X_1$ and $X_2$.
\end{enumerate}

\item 
Let $I$ be an index set and let $M_i$ be an $R$-module for each $i \in I$. Let $M$ be the set of all functions $f:I \to \bigcup_{i \in I} M_i$ such that $f(j) \in M_j$ for each $j \in I$. For $j \in I$, let $\pi_j:M \to M_i$ be the mapping $\pi_j(f) = f(j)$.

Then $M$ is an $R$-module in a natural way, and $\pi_j$ is an $R$-module homomorphism for each $j$.

\begin{enumerate}[label=\alph*.]
    \item Explain why $(M, \pi_j)$ forms a \emph{product} of the $M_i$ in the category $\text{mod}(R)$.

    We usually write $M = \prod_{i \in I} M_i$ for this $R$-module.

    \item Suppose that $I$ is a finite set, let $(\prod_{i \in I} M_i, \pi_i)$ be a product of the $M_i$, and let $(\bigoplus_{i \in I} M_i, \iota_i)$ be a coproduct of the $M_i$. Show that there is an isomorphism $\Phi:\bigoplus_{i \in I} M_i \to \prod_{i \in I} M_i$ of $R$-modules such that for $i,j \in I$ we have $\pi_j \circ \Phi \circ \iota_i = \begin{cases} \text{id} & \text{if } i = j \\ 0 & \text{otherwise} \end{cases}$.
\end{enumerate}

\item (problem removed).

\item
Let $M,N$ be free $R$-modules. Thus there is some set $B$ and function $\beta:B \to M$ such that $M$ is a free $R$-module $\beta:B \to M$, and similarly for $N$.

Prove that $M \oplus N$ is a free $R$-module.

\item 
Let $M$ be an $R$-module, let $I \subseteq R$ be an ideal. Assume that $ax = 0$ for each $a \in I$ and each $x \in M$. Show that $M$ has the structure of an $R/I$-module.

\item
Let $I$ be an ideal of $R$ and let $M$ be an $R$-module.

Let $IM$ be the $R$-submodule of $M$ generated by the set 
$$\{ ax \mid a \in I, x \in M \}.$$

\begin{enumerate}[label=\alph*.]
    \item Prove that the $R$-module $M/IM$ has the structure of an $R/I$-module. (Use the previous question.)

    \item If $M$ is a free $R$-module on $\beta:B \to M$, prove that $M/IM$ is a free $R/I$-module on $\beta' = \pi \circ \beta:B \to M/IM$ where $\pi:M \to M/IM$ is the quotient morphism.
\end{enumerate}

\end{enumerate}

\end{document}

