\documentclass{article}
\usepackage{amsmath}
\usepackage{amsfonts}
\usepackage{amssymb}
\usepackage{amsthm}
\usepackage{tikz}

\title{Problem Set Week 9}
\author{Tufts University \\ Fall 2025 \\ Math 245 \\ Prof. George McNinch}
\date{Due: 2025-11-03}

\newtheorem{problem}{Problem}
\newcommand{\Hom}{\text{Hom}}
\newcommand{\Aut}{\text{Aut}}
\newcommand{\End}{\text{End}}
\newcommand{\Gal}{\text{Gal}}
\newcommand{\Mat}{\text{Mat}}
\newcommand{\GL}{\text{GL}}
\newcommand{\Inn}{\text{Inn}}
\newcommand{\Prop}{\text{Prop}}
\newcommand{\Stab}{\text{Stab}}
\newcommand{\Syl}{\text{Syl}}
\newcommand{\Cl}{\text{Cl}}
\newcommand{\ZZ}{\mathbb{Z}}
\newcommand{\QQ}{\mathbb{Q}}
\newcommand{\RR}{\mathbb{R}}
\newcommand{\CC}{\mathbb{C}}
\newcommand{\mfk}[1]{\mathfrak{#1}}

\begin{document}

\maketitle

Let $ A $ be a commutative ring (with identity).

\begin{enumerate}
  
\item Suppose that $ A $ is an integral domain. Prove that the
  polynomial ring $ A[T] $ is an integral domain.


\item Let $ F $ be a field and let $ f \in F[T] $ be a polynomial of
  degree $ \geq 1 $. An element $ \alpha \in F $ is a root of $ f $ if
  $ f(\alpha) = 0 $. Recall that $ f(\alpha) = \varphi(f) $ is the
  result of applying the ring homomorphism $ \varphi : F[T] \to F $
  determined by the properties: $ \varphi_{|F} = \text{id} $ and $
  \varphi(T) = \alpha $.

    \begin{enumerate}
        \item Prove that if $ \deg f = 2 $ or $ \deg f = 3 $, then $ f $ is irreducible if and only if $ f $ has no root in $ F $.
        \item Prove that in general, there are reducible polynomials with no root in $ F $ provided that $ \deg f \geq 4 $.
    \end{enumerate}


  \item Let $ d \in \ZZ $, let $ f = T^2 - d \in \ZZ[T] $, and
    consider the ring $ \ZZ[\sqrt{d}] \subset \CC $. Assume that $
    \forall a \in \ZZ, d \neq a^2 $. In that case, one knows that $
    \sqrt{d} \not\in \QQ $.

    \begin{enumerate}
        \item Explain why $ f = T^2 - d $ is irreducible in $ \QQ[T] $ and prove that $ \QQ[T]/\langle T^2 - d \rangle \cong \QQ[\sqrt{d}] $.
        \item Explain why $ f = T^2 - d $ is irreducible in $ \ZZ[T] $ and prove that $ \ZZ[T]/\langle T^2 - d \rangle $ is a free $ \ZZ $-module of rank 2.
        \item Prove that $ \QQ(\sqrt{d}) $ may be identified with the field of fractions of $ \ZZ[\sqrt{d}] $.
    \end{enumerate}


  \item Let $ A = \ZZ[i] $.  Then $ A $ is a Euclidean Domain with
    norm given by $ N(a + bi) = a^2 + b^2 $ for $ a,b \in \ZZ $ -- you
    are free to use this; for a reference see [Dummit-Foote,
    \S 8.1 example 3 p. 272].

    \begin{enumerate}
        \item Show that $ N(\alpha) = \pm 1 \iff \alpha \in A^\times $.
        \item Show for any integer $ n $ that $ A/A \cdot n $ is a ring with
        $ n^2 $ elements.
        \item Show that $ 7 $ is irreducible in $ A $. 
            \textit{Hint:} Is it possible to write $ 7 = a^2 + b^2 $ for $ a,b \in \ZZ $?
            Conclude that $ A/A \cdot 7 $ is a field with $ 49 $ elements.
    \end{enumerate}


  \item Again let $ A = \ZZ[i] $ and keep the notation $ N $ for the
    norm on $ A $.

    \begin{enumerate}
        \item Show that if $ N(\alpha) $ is a prime integer, then $ \alpha $ is irreducible and hence prime.
        \item Show that $ 2 + i $ and $ 2 - i $ are both prime in $ A $.
        \item Prove that $ A/A \cdot (2 + i) $ and $ A/A \cdot (2 - i) $ are both fields, each with $ 5 $ elements.
        \item Prove that 
            $ A/A \cdot 5 \cong A/A \cdot (2 + i) \times A/A \cdot (2 - i) $.
            (Use the Chinese Remainder Theorem.)
    \end{enumerate}


  \item Let $ K $ be an uncountable field (e.g., $ K = \RR $ or $ K =
    \CC $).  Prove that $ K(T) $ contains a $ K $-linearly independent
    subset which is uncountable.

    \textit{Hint:} Consider the set $ \{ \frac{1}{T - \alpha} \mid \alpha \in K \} $.


  \item Let $ K $ be a field, and let $ p_1,p_2,\ldots,p_r \in K[T] $
    be pairwise non-associate irreducible polynomials for some $ r \in
    \mathbb{N} $. Let $ S $ be a new polynomial variable and consider
    the polynomial $ f = S^2 - p_1 \cdots p_r \in K[S,T] $.

    \begin{enumerate}
        \item Prove that if $ r > 0 $, then $ f $ is irreducible in $ K(T)[S] $.

            \textit{Hint:} Argue that $ f $ has no root in $ K(T) $ just as in the classical proof that $ \sqrt{2} $ is irrational.
        \item Explain how to deduce that $ f $ is irreducible -- and hence prime -- in $ K[T,S] $.
        \item Let $ A = K[T,S]/\langle f \rangle $ and write $ s $ for the image of $ S $ in $ A $. Prove that the integral domain $ A $ is a free $ K[T] $-module on a basis $ \{1,s\} $.
        \item Prove that $ K(T)[S]/\langle f \rangle $ may be identified with the field of fractions of $ K[S,T]/\langle f \rangle $.
    \end{enumerate}


\end{enumerate}

\end{document}
